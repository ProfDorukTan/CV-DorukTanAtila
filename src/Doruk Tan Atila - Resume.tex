\documentclass{ExpressiveResume}

\begin{document}

%--------------------------------------------------------------------------------------%
\resumeheader[
    firstname=              Doruk Tan,
    middleinitial=          ,
    lastname=               Atila,
    email=                  doruktanatila4@gmail.com,
    phone=                  773-313-1169,
    linkedin=               doruktanatila,
    github=                 ProfDorukTan,
    city=                   Manchester,
    state=                  UK,
    qrcode=                 ./images/Linkedin_QR.png,
    fixobjectivespacing=    true
]


%--------------------------------------------------------------------------------------%
\objective{Driven Electronic Engineer with expertise in \textbf{embedded
        systems} and \textbf{hardware design}. Passionate about developing
    innovative solutions across diverse applications, with professional
    experience in \textbf{aerospace electronics}.}



%--------------------------------------------------------------------------------------%
\section{Education}

\education{Bachelor of Engineering}{The University of Manchester,
    Manchester, UK}{Sept 2021}{December 2024}{1st Class Honours}{
    Digital Design, Computer Architecture, VLSI Design,
    Microcontroller Engineering 1\&2, Embedded Systems
}

\education{High School Diploma}{The American Robert College of Istanbul,
    Turkey}{Sept 2016}{June 2021}{88.03/100}{Electronics, AP Physics C,
    AP Calculus BC}


%--------------------------------------------------------------------------------------%
\section{Publications}

\publication{A Novel Approach to Modelling of MOS Devices for
    Simulation}{2024}{GitHub}{
    Developed a simulation tool for analyzing MOSFET behavior,
    incorporating quantum-mechanical semiconductor physics.
}

\publication{Organic Phototransistors as Artificial Synapses for
    Neuromorphic Systems}{2024}{Sensors and Actuators A: Physical}{
    Researched novel organic phototransistors with potential for
    neuromorphic computing applications.
}



%--------------------------------------------------------------------------------------%
\section{Projects}

\project{Wireless Video and Audio Broadcasting System on Ultra96-V2
    FPGA}{Oct 2024}{PRESENT}{Designed and developed a hardware solution for
    wireless video and audio transmission using the Ultra96-V2 FPGA board.
    The design emphasizes high-performance compression algorithms and
    real-time data transmission for multimedia applications.}{FPGA, Verilog,
    Xilinx, Xilinx UltraScale+, Audio/Video Compression}

\project{Automatic Plant Monitoring \& Watering System}{Oct 2024}{Oct
    2024}{Developed a low-power STM32F401RE-based system to monitor soil
    moisture and activate water pumps. Utilized RTC for periodic
    wake-ups, achieving
    ~99.99\% power reduction compared to continuous operation.
}{Power Optimization, Microcontrollers, STM32, C, Embedded Systems,
    RTC Programming}

\project{BNO055 Sensor Implementation Using STM HAL}{Oct 2024}{Oct
    2024}{Developed drivers for the Bosch BNO055 IMU sensor and integrated
    it with STM32F401RE using STM HAL. The sensor provides data for a
    secondary rocket controller, with an optimized I2C protocol for
    low-latency communication.
}{I2C, RTL Coding, Embedded Systems, IMU, STM HAL}

\project{MOSFET Modelling and Simulation}{Sept 2023}{May 2024}{Derived a
    MOSFET model from quantum-mechanical semiconductor physics foundations
    and built a standalone C++ program capable of simulating the output and
    transfer curves of a MOSFET with manufacturing parameters.
}{C++, Qt, MOSFET, Solid State Physics, Modeling, Simulation, Semiconductor Device, OFET}

\project{Organic Phototransistor Array for Infrared Imaging}{May
    2024}{May 2024}{Designed an organic phototransistor array for infrared
    imaging using a conjugated polymer from literature, incorporating filter
    arrays to detect NIR, SWIR, MWIR, and broadband wavelengths for precise
    detection of specific infrared bands.
}{Sensor Design, Signal Conditioning, Imaging Sensors}

\project{Cache Simulator}{Apr 2024}{Apr 2024}{Developed a cache
    simulator in C to evaluate and compare different cache configurations
    using real memory trace files. The simulator provides insights into
    cache performance metrics such as hit/miss rates and access latency,
    assisting in optimizing memory hierarchy designs.}{C, Cache, Simulation, Memory Systems}

\project{Concurrent Systems}{Nov 2023}{Nov 2023}{Coded a simulation of a
    four-by-one hundred-meter sprint relay race. The project involved
    creating threads, synchronizing them, and ensuring thread safety.
}{C++, Multithreading, Simulation}

\project{Microcontroller-Peripheral Communication I2C/SPI}{Nov 2023}{Nov
    2023}{Programmed STM32F401RE microcontroller using low-layer C libraries
    to enable data transfer between the controller and peripherals.
}{Embedded Systems, C, SPI, I2C, Keil uVision}

\project{Line Following Buggy (STM32F401RE)}{Sep 2022}{June 2023}{Programmed
    STM32F401RE microcontroller to create a line-following buggy. The system
    gathered data from six light sensors, determined the line's position,
    and adjusted the motor speed to center the line using PID control.
}{Embedded Systems, C++, STM32, Algorithms}

\project{VLSI Design}{May 2023}{May 2023}{Designed a 250nm technology
    CMOS schematic and layout of a circuit. Optimized the circuit for given
    speed/power requirements and passed DRC and LVS checks. The circuit
    implemented a logic boolean expression with 3 inputs.
}{Tanner EDA, Calibre, CMOS, Very-Large-Scale Integration (VLSI)}

\project{Line Following Buggy (Nvidia
    Jetson Nano)}{Apr 2023}{Apr 2023}{Programmed Nvidia
    Jetson Nano for a line-following buggy. The algorithm captured
    front-facing video, processed the image to detect line contours, and
    utilized PD motor control to align the line as closely as possible to
    the middle of the buggy.
}{Python, Robot Operating System (ROS), Embedded Systems, Image Processing}

\project{VHDL Stopwatch}{Nov 2022}{Nov 2022}{Developed a stopwatch on
    FPGA using VHDL and finite state machines for timing control.
}{FPGA, VHDL, Finite State Machine}

\project{Building a Cargo Carrying Drone}{Jan 2018}{Jan 2019}{Built a
    quadcopter for a competition, coding two Arduino boards for wireless
    control of an electromagnet to carry/drop cargo. Team placed 1st in
    Turkey.}{Drones, Arduino, Microcontrollers,
    Embedded Systems, Python, Cabling, Problem Solving}


%--------------------------------------------------------------------------------------%
\section{Work Experience}

\workexperience{Data Analyst/Business Developer}{May 2022}{June
    2023}{Arcanor - Istanbul, Turkey}{Optimized SQL queries and
    developed an algorithm to query cross-country mobility data.
    Coordinated various tasks as part of a startup team,
    working extensively in product development, documentation, website
    management, Google services, and translation.
}{Data Analysis, Project Management, SQL, Startup Management, Google
    Services}

\workexperience{Intern}{June 2019}{July 2019}{Softtech Software
    Technologies R\&D - Istanbul, Turkey}{Wrote and implemented Python
    code for automated and manual drone controlling. Trained a pattern
    recognition algorithm in Python to
    identify a medical disease, working with software development team
    members.
}{Simulation, Linux (Ubuntu), Embedded Systems, Python, Machine
    Learning}


%--------------------------------------------------------------------------------------%
\section{Skills}
\begin{itemize}[leftmargin=12pt]
    \item \textbf{Programming Languages}: Verilog, VHDL, Assembly, C,
          C++, Python, CUDA, MATLAB with Simulink, Mbed, SQL
    \item \textbf{Software Tools}: Xilinx, STM32CubeIDE, NI LabVIEW,
          Tanner EDA Tools, Calibre Verification, Altium Designer, Keil
          uVision
    \item \textbf{Technical Skills}: High speed \& analog PCB design, PVD,
          Spin-coating, Photolithography, Electrical
          Characterizations\end{itemize}



%--------------------------------------------------------------------------------------%
\section{Certifications}
\achievement{
    Accelerating CUDA C++ Applications with Concurrent Streams \hfill
    \textsc{Feb 2023}
}
\achievement{
    Fundamentals of Accelerated Computing with CUDA C/C++ \hfill
    \textsc{Jan 2023}
}
\end{document}